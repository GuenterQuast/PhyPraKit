%% Generated by Sphinx.
\def\sphinxdocclass{report}
\documentclass[letterpaper,10pt,english]{sphinxmanual}
\ifdefined\pdfpxdimen
   \let\sphinxpxdimen\pdfpxdimen\else\newdimen\sphinxpxdimen
\fi \sphinxpxdimen=.75bp\relax

\usepackage[utf8]{inputenc}
\ifdefined\DeclareUnicodeCharacter
 \ifdefined\DeclareUnicodeCharacterAsOptional\else
  \DeclareUnicodeCharacter{00A0}{\nobreakspace}
\fi\fi
\usepackage{cmap}
\usepackage[T1]{fontenc}
\usepackage{amsmath,amssymb,amstext}
\usepackage{babel}
\usepackage{times}
\usepackage[Bjarne]{fncychap}
\usepackage{longtable}
\usepackage{sphinx}

\usepackage{geometry}
\usepackage{multirow}
\usepackage{eqparbox}

% Include hyperref last.
\usepackage{hyperref}
% Fix anchor placement for figures with captions.
\usepackage{hypcap}% it must be loaded after hyperref.
% Set up styles of URL: it should be placed after hyperref.
\urlstyle{same}

\addto\captionsenglish{\renewcommand{\figurename}{Fig.}}
\addto\captionsenglish{\renewcommand{\tablename}{Table}}
\addto\captionsenglish{\renewcommand{\literalblockname}{Listing}}

\addto\extrasenglish{\def\pageautorefname{page}}

\setcounter{tocdepth}{1}



\title{PhyPraKit Documentation}
\date{May 21, 2017}
\release{1.0}
\author{Günter Quast}
\newcommand{\sphinxlogo}{}
\renewcommand{\releasename}{Release}
\makeindex

\begin{document}

\maketitle
\sphinxtableofcontents
\phantomsection\label{\detokenize{index::doc}}



\chapter{Übersicht:}
\label{\detokenize{index:ubersicht}}\label{\detokenize{index:dokumentation-von-phyprakit}}
PhyPraKit ist eine Sammlung nützlicher Funktionen in der Sprache
\sphinxtitleref{python (vers. 2.7)} zur Visualisierung und Auswertung von Daten
in den physikalischen Praktika.
Beispiele illustrieren jeweils die Anwendung.
\begin{quote}

\sphinxtitleref{Version der Dokumentation vom Mai 2017}
\end{quote}


\chapter{Indices and tables}
\label{\detokenize{index:indices-and-tables}}\begin{itemize}
\item {} 
\DUrole{xref,std,std-ref}{genindex}

\item {} 
\DUrole{xref,std,std-ref}{modindex}

\item {} 
\DUrole{xref,std,std-ref}{search}

\end{itemize}


\chapter{\sphinxstylestrong{Darstellung und Auswertung von Messdaten}}
\label{\detokenize{index:darstellung-und-auswertung-von-messdaten}}
In allen Praktika zur Physik werden Methoden zur Darstelllung
und Auswertung von Messdaten benötigt. Die Script- und
Programmiersprache \sphinxtitleref{python} mit den Zusatzpaketen
\sphinxtitleref{numpy} und \sphinxtitleref{matplotlib} ist ein universelles
Werkzeug, um die Wiederholbarkeit und Reprodzierbarkeit von
Datenauswertungen zu gewährleiseten.

In der Veranstaltung \sphinxquotedblleft{}Computergestützte Datenauswertung\sphinxquotedblright{}
(\sphinxurl{http://www.ekp.kit.edu/~quast/CgDA}), die im neuen Studienplan
für den Bachelorstudiengang Physik am KIT seit dem Sommersemester
2016 angeboten wird, werden Methoden und Software zur grafischen Darstellung von Daten, deren Modellierung und Auswertung eingeführt.

Die folgen Links erlauben einen schnellen Überblick über die Inhalte
der Vorlesung und den Beispeielen aus den Übungen:
\begin{itemize}
\item {} \begin{description}
\item[{Zusammenfassung der Vorlesung und Dokumentation der Code-Beispiele}] \leavevmode
\sphinxurl{http://www.ekp.kit.edu/~quast/CgDA/CgDA-html/CgDA\_ZusFas.html}

\end{description}

\item {} \begin{description}
\item[{Installation der Software auf verschiedenen Platformen}] \leavevmode\begin{itemize}
\item {} 
Dokumentation in html:
\sphinxurl{http://www.ekp.kit.edu/~quast/CgDA/CgDA-SoftwareInstallation-html}

\item {} 
Dokumentation in pdf:
\sphinxurl{http://www.ekp.kit.edu/~quast/CgDA/CgDA-SoftwareInstallation.pdf}

\item {} 
Softwarepakete:
\sphinxurl{http://www.ekp.kit.edu/~quast/CgDA/Software}

\end{itemize}

\end{description}

\end{itemize}

Speziell für das \sphinxquotedblleft{}Praktikum zur klassischen Physik\sphinxquotedblright{} finden sich eine
kurze Einführung
(\sphinxurl{http://www.ekp.kit.edu/~quast/CgDA/PhysPrakt/CgDA\_APraktikum.pdf})
sowie die hier dokumentierten einfachen Beispiele als Startpunkt für
eigene Auswertungen
(\sphinxurl{http://www.ekp.kit.edu/~quast/CgDA/PhysPrakt/}).

Die vorliegende Sammlung von Funktionen im Paket \sphinxtitleref{PhyPraKit} enthält
Funktionen zum Einlesen von Daten aus diversen Quellen, zur
Datenvisualisierung, Signalbearbeitung und statistischen Datenauswertung
und Modellanpassung sowie Werkzeuge zur Erzeugung simulierter Daten.
Dabei wurde absichlich Wert auf eine einfache, die Prinzipien
unterstreichnede Codierung gelegt und nicht der möglichst effizienten
bzw. allgemeinensten Implementierung der Vorzug gegeben.


\chapter{Dokumentation der Beispiele}
\label{\detokenize{index:dokumentation-der-beispiele}}\begin{description}
\item[{\sphinxtitleref{PhyPraKit.py} ist ein Paket mit nützlichen Hilfsfunktionen}] \leavevmode\begin{quote}

zum import in eigene Beispielen mittels:

\begin{sphinxVerbatim}[commandchars=\\\{\}]
\PYG{k+kn}{import} \PYG{n+nn}{PhyPraKit} \PYG{k}{as} \PYG{n+nn}{ppk}
\end{sphinxVerbatim}

oder:

\begin{sphinxVerbatim}[commandchars=\\\{\}]
\PYG{k+kn}{from} \PYG{n+nn}{PhyPraKit} \PYG{k}{import} \PYG{o}{.}\PYG{o}{.}\PYG{o}{.}
\end{sphinxVerbatim}
\end{quote}
\begin{itemize}
\item {} 
\sphinxtitleref{test\_readColumnData.py} ist ein Beispiel zum
Einlesen von Spalten aus Textdateien; die zugehörigen
\sphinxstyleemphasis{Metadaten} können ebenfalls an das Script
übergeben werden und stehen so bei der Auswsertung zur Verfügung.

\item {} 
\sphinxtitleref{test\_readtxt.py} liest Ausgabedateien im allgemeinem  .txt-Format
\begin{itemize}
\item {} 
Entfernen aller ASCII-Sonderzeichen außer dem Spalten-Trenner

\item {} 
Ersetzen des deutschen Dezimalkommas durch Dezimalpunkt

\end{itemize}

\item {} 
\sphinxtitleref{test\_readPicoScope.py} liest Ausgabedateien von PicoScpe im .txt-Format

\item {} 
\sphinxtitleref{test\_labxParser.py} liest Ausgabedateien von Leybold
CASSY im .labx-Format. Die Kopfzeilen und Daten von Messreihen
werden als Listen in python zur Verfügung gestellt.

\item {} 
\sphinxtitleref{test\_AutoCorrelation.py} liest die \sphinxtitleref{Datei AudioData.csv} und führt
ein Analyse der Autokorrelation zur Frequenzbestimmung durch.

\item {} 
\sphinxtitleref{test\_kRegression.py} dient zur Anpassung einer Geraden an
Messdaten mit Fehlern in Ordinaten- und Abszissenrichtung und mit allen
Messpunkten gemeinsamen (d. h. korrelierten) relativen oder absoluten
systematischen Fehlern mit dem Paket \sphinxtitleref{kafe}.

\item {} 
\sphinxtitleref{test\_linRegression.py} ist eine einfachere Version mit
\sphinxtitleref{python}-Bordmitteln zur Anpassung einer Geraden an
Messdaten mit Fehlern in Ordinaten- und Abszissenrichtung.
Korrelierte Unsicherheiten werden nicht unterstützt.

\item {} 
\sphinxtitleref{test\_kFit.py} ist eine verallgemeinerte Version von \sphinxtitleref{test\_kRegression}
und dient zur Anpassung einer beliebigen Funktion an Messdaten mit
Fehlern in Ordinaten- und Abszissenrichtung und mit allen Messpunkten
gemeinsamen (d. h. korrelierten) relativen oder absoluten systematischen
Fehlern mit dem Paket \sphinxtitleref{kafe}.

\item {} 
\sphinxtitleref{test\_Histogram.py}  ist ein Beispiel zur Darstellung und
statistischen Auswertung von Häufigkeitsverteilungen (Histogrammen) in
einer und zwei Dimensionen.

\item {} 
\sphinxtitleref{test\_generateXYata.py} zeigt, wie man mit Hilfe von Zufallszahlen
\sphinxquotedblleft{}künstliche Daten\sphinxquotedblright{} zur Veranschaulichung oder zum Test von Methoden
zur Datenauswertung erzeugen kann.

\end{itemize}

Weitere python-Skripte als Beisepiele zur Anwendung von Modulen
in \sphinxtitleref{PhyPraKit}:
\begin{itemize}
\item {} 
\sphinxtitleref{kfitf.py} ist ein Kommandozeilen-Werkzeug, mit dem man komfortabel
Anpassungen ausführen kann, bei denen Daten und Fit-Funktion in
einer einzigen Datei angegeben werden. Beispiele finden sich
in den Dateien mit der Endung \sphinxtitleref{.fit}.

\item {} 
\sphinxtitleref{Beispiel\_Drehpendel.py} demonstriert die Analyse von am Drehpendel
mit CASSY aufgenommenen Daten. Enthalten sind einfache Funktionen zum
Filtern und bearbeiten der Daten, Suche nach Extrema, zur Anpassung
einer Einhüllenden, zur diskreten Fourier-Transformation und zur
Interpolation von Messdaten mit kubischen Spline-Funktionen.

\item {} 
\sphinxtitleref{Beispiel\_Hysterese.py} demonstriert die Analyse von Daten,
die mit einem USB-Oszilloskop der Marke \sphinxtitleref{PicoScope} am
Versuch zur Hysterese aufgenommen wurden. Die aufgezeichneten Werte
für Strom und B-Feld werden in einen Zweig für steigenden und
fallenden Strom aufgeteilt, mit Hilfe von kubischen Splines
interpoliert und dann integriert.

\item {} 
\sphinxtitleref{Beispiel\_Wellenform.py}  zeigt eine typische Auswertung
periodischer Daten am Beispiel der akustischen Anregung eines
Metallstabs. Genutzt werden Fourier-Transformation und
eine Suche nach charakteristischen Extrema. Die Zeitdifferenzen
zwischen deren Auftreten im Muster werden bestimmt, als
Häufgkeitsverteilung dargestellt und die Verteilungen statistisch
ausgewertet.

\item {} 
\sphinxtitleref{Beispiel\_GammaSpektroskopie.py} liest mit dem Vielkanalanalysator
des CASSY-Systems aufgezeichnete Spekren aus einer im .labx-Format
gespeicherten Datei ein.

\end{itemize}

\end{description}

Die übrigen \sphinxtitleref{python}-Scripte im Verzeichnis wurden zur Erstellung
der in der einführenden Vorlesung gezeigten Grafiken verwendet.

Für die \sphinxstylestrong{Erstellung von Protokollen} mit Tabellen, Grafiken und Formeln
bietet sich das Textsatz-System \sphinxtitleref{LaTeX} an. Die Datei
\sphinxtitleref{Protokollvorlage.zip}
enthält eine sehr einfach gehaltene Vorlage, die für eigene Protokolle
verwendet werden kann. Eine sehr viel umfangreichere Einführung sowie
ein ausführliches Beispiel bietet die Fachschaft Physik unter dem
Link \sphinxurl{https://fachschaft.physik.kit.edu/drupal/content/latex-vorlagen}


\chapter{Modul-Dokumentation}
\label{\detokenize{index:module-PhyPraKit}}\label{\detokenize{index:modul-dokumentation}}\index{PhyPraKit (module)}\begin{description}
\item[{\sphinxstylestrong{PhyPraKit}  }] \leavevmode
a collection of tools for data handling, visualisation and analysis 
in Physics Lab Courses, recommended for \sphinxquotedblleft{}Physikalisches Praktikum am KIT\sphinxquotedblright{}

\end{description}
\phantomsection\label{\detokenize{index:module-PhyPraKit.PhyPraKit}}\index{PhyPraKit.PhyPraKit (module)}
\sphinxstylestrong{PhyPraKit}  for Data Handling, Visualisation and Analysis
\begin{quote}

contains the follwoing functions:
\begin{quote}
\begin{enumerate}
\item {} 
Data input:

\end{enumerate}
\begin{itemize}
\item {} 
readColumnData() read data and meta-data from text file

\item {} 
readCSV()        read data in csv-format from file with header

\item {} 
readtxt()        read data in \sphinxquotedblleft{}txt\sphinxquotedblright{}-format from file with header

\item {} 
readPicoScope()  read data from PicoScope

\item {} 
readCassy()      read CASSY output file in .txt format

\item {} 
labxParser()     read CASSY output file, .labx format

\item {} 
writeCSV()       write data in csv-format (opt. with header)

\end{itemize}
\begin{enumerate}
\setcounter{enumi}{1}
\item {} 
signal processing:

\end{enumerate}
\begin{itemize}
\item {} 
offsetFilter()     subtract an offset in array a

\item {} 
meanFilter()       apply sliding average to smoothen data

\item {} 
resample()         average over n samples

\item {} 
Fourier\_fft()      fast Fourier transformation of an array

\item {} 
FourierSpectrum()  Fourier transformation of an array 
\sphinxcode{(slow, preferably use fft version)}

\item {} \begin{description}
\item[{simplePeakfinder() find peaks and dips in an array comparing }] \leavevmode
neighbouring samples  \sphinxcode{use convolutionPeakfinder}

\end{description}

\item {} 
convolutionPeakfinder() find peaks and dips in an array

\item {} 
atocorrelate()     autocorrelation function

\end{itemize}
\begin{enumerate}
\setcounter{enumi}{2}
\item {} 
statistics:

\end{enumerate}
\begin{itemize}
\item {} 
wmean()  weighted mean

\end{itemize}
\begin{enumerate}
\setcounter{enumi}{3}
\item {} 
histograms tools:

\end{enumerate}
\begin{itemize}
\item {} 
barstat()   statistical information bar chart

\item {} 
nhist()    histogram plot based on np.historgram() and plt.bar()
\sphinxcode{use matplotlib.pyplot.hist() instead}

\item {} 
histstat() statistical information from 1d-histogram

\item {} 
nhist2d()  2d-histotram plot based on np.histrogram2d, plt.colormesh()
\sphinxcode{use matplotlib.pyplot.hist2d() instead}

\item {} 
hist2dstat() statistical information from 1d-histogram

\item {} 
profile2d()  \sphinxquotedblleft{}profile plot\sphinxquotedblright{} for 2d data

\item {} 
chi2p\_indep2d() chi2 test on independence of data

\end{itemize}
\begin{enumerate}
\setcounter{enumi}{4}
\item {} 
linear regression:

\end{enumerate}
\begin{itemize}
\item {} 
linRegression()    linear regression, y=ax+b, with analytical formula

\item {} 
linRegressionXY()  linear regression, y=ax+b, with x and y errors 
\sphinxcode{! deprecated, use {}`odFit{}` with linear model instead}

\item {} 
odFit()            fit function with x and y errors (scipy ODR)

\item {} 
kRegression()      regression, y=ax+b, with x-, y- and correlated errors
\sphinxcode{! deprecated, use {}`kFit{}` with linear model instead}

\item {} 
kFit()             fit function with x-, y- and correlated errors (kafe)

\end{itemize}
\begin{enumerate}
\setcounter{enumi}{5}
\item {} 
simulated data with MC-method:

\end{enumerate}
\begin{itemize}
\item {} 
smearData()          add random deviations to input data

\item {} 
generateXYdata()     generate simulated data

\end{itemize}
\end{quote}
\end{quote}
\index{FourierSpectrum() (in module PhyPraKit.PhyPraKit)}

\begin{fulllineitems}
\phantomsection\label{\detokenize{index:PhyPraKit.PhyPraKit.FourierSpectrum}}\pysiglinewithargsret{\sphinxcode{PhyPraKit.PhyPraKit.}\sphinxbfcode{FourierSpectrum}}{\emph{t}, \emph{a}, \emph{fmax=None}}{}~\begin{description}
\item[{Fourier transform of amplitude spectrum a(t), for equidistant sampling times}] \leavevmode
(a simple implementaion for didactical purpose only, 
consider using \sphinxcode{Fourier\_fft()} )
\begin{quote}
\begin{description}
\item[{Args:}] \leavevmode\begin{itemize}
\item {} 
t: np-array of time values

\item {} 
a: np-array amplidude a(t)

\end{itemize}

\item[{Returns:}] \leavevmode\begin{itemize}
\item {} 
arrays freq, amp: frequencies and amplitudes

\end{itemize}

\end{description}
\end{quote}

\end{description}

\end{fulllineitems}

\index{Fourier\_fft() (in module PhyPraKit.PhyPraKit)}

\begin{fulllineitems}
\phantomsection\label{\detokenize{index:PhyPraKit.PhyPraKit.Fourier_fft}}\pysiglinewithargsret{\sphinxcode{PhyPraKit.PhyPraKit.}\sphinxbfcode{Fourier\_fft}}{\emph{t}, \emph{a}}{}
Fourier transform of the amplitude spectrum a(t)
\begin{description}
\item[{method: }] \leavevmode
uses \sphinxtitleref{numpy.fft} and \sphinxtitleref{numpy.fftfreq}; 
output amplitude is normalised to number of samples;
\begin{description}
\item[{Args:}] \leavevmode\begin{itemize}
\item {} 
t: np-array of time values

\item {} 
a: np-array amplidude a(t)

\end{itemize}

\item[{Returns:}] \leavevmode\begin{itemize}
\item {} 
arrays f, a\_f: frequencies and amplitudes

\end{itemize}

\end{description}

\end{description}

\end{fulllineitems}

\index{autocorrelate() (in module PhyPraKit.PhyPraKit)}

\begin{fulllineitems}
\phantomsection\label{\detokenize{index:PhyPraKit.PhyPraKit.autocorrelate}}\pysiglinewithargsret{\sphinxcode{PhyPraKit.PhyPraKit.}\sphinxbfcode{autocorrelate}}{\emph{a}}{}
calculate autocorrelation function of array

method: for array of length l, calulate 
a{[}0{]}=sum\_(i=0)\textasciicircum{}(l-1) a{[}i{]}*{[}i{]}
a{[}i{]}= 1/a{[}0{]} * sum\_(k=0)\textasciicircum{}(l-i) a{[}i{]} * a{[}i+k-1{]} for i=1,l-1
uses inner product from numpy
\begin{description}
\item[{Args:}] \leavevmode\begin{itemize}
\item {} 
a: np-array

\end{itemize}

\item[{Returns }] \leavevmode\begin{itemize}
\item {} 
np-array of len(a)

\end{itemize}

\end{description}

\end{fulllineitems}

\index{barstat() (in module PhyPraKit.PhyPraKit)}

\begin{fulllineitems}
\phantomsection\label{\detokenize{index:PhyPraKit.PhyPraKit.barstat}}\pysiglinewithargsret{\sphinxcode{PhyPraKit.PhyPraKit.}\sphinxbfcode{barstat}}{\emph{bincont}, \emph{bincent}, \emph{pr=True}}{}
statistics from a bar chart (histogram) 
with given bin contents and bin centres
\begin{description}
\item[{Args:}] \leavevmode\begin{itemize}
\item {} 
bincont: array with bin content

\item {} 
bincent: array with bin centres

\end{itemize}

\item[{Returns:}] \leavevmode\begin{itemize}
\item {} 
float: mean, sigma and sigma on mean

\end{itemize}

\end{description}

\end{fulllineitems}

\index{chi2p\_indep2d() (in module PhyPraKit.PhyPraKit)}

\begin{fulllineitems}
\phantomsection\label{\detokenize{index:PhyPraKit.PhyPraKit.chi2p_indep2d}}\pysiglinewithargsret{\sphinxcode{PhyPraKit.PhyPraKit.}\sphinxbfcode{chi2p\_indep2d}}{\emph{H2d}, \emph{bcx}, \emph{bcy}, \emph{pr=True}}{}
perform a chi2-test on independence of x and y
\begin{description}
\item[{Args:}] \leavevmode\begin{itemize}
\item {} 
H2d: histogram array (as returned by histogram2d)

\item {} 
bcx: bin contents x

\item {} 
bcy: bin contents y

\end{itemize}

\item[{Returns:}] \leavevmode\begin{itemize}
\item {} 
float: p-value w.r.t. assumption of independence

\end{itemize}

\end{description}

\end{fulllineitems}

\index{convolutionPeakfinder() (in module PhyPraKit.PhyPraKit)}

\begin{fulllineitems}
\phantomsection\label{\detokenize{index:PhyPraKit.PhyPraKit.convolutionPeakfinder}}\pysiglinewithargsret{\sphinxcode{PhyPraKit.PhyPraKit.}\sphinxbfcode{convolutionPeakfinder}}{\emph{a}, \emph{width=10}, \emph{th=0.1}}{}~\begin{description}
\item[{find positions of all Peaks and Dips in data }] \leavevmode
(simple version for didactical purpose, 
consider using \sphinxcode{scipy.signal.find\_peaks\_cwt()} )

\item[{method: }] \leavevmode
convolute array a with signal template of given width and
return extrema of convoluted signal, i.e. places where 
template matches best

\item[{Args:}] \leavevmode\begin{itemize}
\item {} 
a: array-like, input data

\item {} 
width: int, width of signal to search for

\item {} 
th: float, relative threshold for peaks above minimum

\end{itemize}

\item[{Returns:}] \leavevmode\begin{itemize}
\item {} 
pidx: list, indices (in original array) of peaks

\end{itemize}

\end{description}

\end{fulllineitems}

\index{generateXYdata() (in module PhyPraKit.PhyPraKit)}

\begin{fulllineitems}
\phantomsection\label{\detokenize{index:PhyPraKit.PhyPraKit.generateXYdata}}\pysiglinewithargsret{\sphinxcode{PhyPraKit.PhyPraKit.}\sphinxbfcode{generateXYdata}}{\emph{xdata}, \emph{model}, \emph{sx}, \emph{sy}, \emph{mpar=None}, \emph{srelx=None}, \emph{srely=None}, \emph{xabscor=None}, \emph{yabscor=None}, \emph{xrelcor=None}, \emph{yrelcor=None}}{}
Generate measurement data according to some model
assumes xdata is measured within the given uncertainties; 
the model function is evaluated at the assumed \sphinxquotedblleft{}true\sphinxquotedblright{} values 
xtrue, and a sample of simulated measurements is obtained by 
adding random deviations according to the uncertainties given 
as arguments.
\begin{description}
\item[{Args:}] \leavevmode\begin{itemize}
\item {} 
xdata:  np-array, x-data (independent data)

\item {} 
model: function that returns (true) model data (y-dat) for input x

\item {} 
mpar: list of parameters for model (if any)

\end{itemize}

\item[{the following are single floats or arrays of length of x}] \leavevmode\begin{itemize}
\item {} 
sx: gaussian uncertainty(ies) on x

\item {} 
sy: gaussian uncertainty(ies) on y

\item {} 
srelx: relative gaussian uncertainty(ies) on x

\item {} 
srely: relative gaussian uncertainty(ies) on y

\end{itemize}

\item[{the following are common (correlated) systematic uncertainties}] \leavevmode\begin{itemize}
\item {} 
xabscor: absolute, correlated error on x

\item {} 
yabscor: absolute, correlated error on y

\item {} 
xrelcor: relative, correlated error on x

\item {} 
yrelcor: relative, correlated error on y

\end{itemize}

\item[{Returns:}] \leavevmode\begin{itemize}
\item {} 
np-arrays of floats:
\begin{itemize}
\item {} 
xtrue: true x-values

\item {} 
ytrue: true value = model(xtrue)

\item {} 
ydata:  simulated data

\end{itemize}

\end{itemize}

\end{description}

\end{fulllineitems}

\index{hist2dstat() (in module PhyPraKit.PhyPraKit)}

\begin{fulllineitems}
\phantomsection\label{\detokenize{index:PhyPraKit.PhyPraKit.hist2dstat}}\pysiglinewithargsret{\sphinxcode{PhyPraKit.PhyPraKit.}\sphinxbfcode{hist2dstat}}{\emph{H2d}, \emph{xed}, \emph{yed}, \emph{pr=True}}{}
calculate statistical information from 2d Histogram
\begin{description}
\item[{Args:}] \leavevmode\begin{itemize}
\item {} 
H2d: histogram array (as returned by histogram2d)

\item {} 
xed: bin edges in x

\item {} 
yed: bin edges in y

\end{itemize}

\item[{Returns:}] \leavevmode\begin{itemize}
\item {} 
float: mean x

\item {} 
float: mean y

\item {} 
float: variance x

\item {} 
float: variance y

\item {} 
float: covariance of x and y

\item {} 
float: correlation of x and y

\end{itemize}

\end{description}

\end{fulllineitems}

\index{histstat() (in module PhyPraKit.PhyPraKit)}

\begin{fulllineitems}
\phantomsection\label{\detokenize{index:PhyPraKit.PhyPraKit.histstat}}\pysiglinewithargsret{\sphinxcode{PhyPraKit.PhyPraKit.}\sphinxbfcode{histstat}}{\emph{binc}, \emph{bine}, \emph{pr=True}}{}
calculate mean of a histogram with bincontents binc and bin edges bine
\begin{description}
\item[{Args:}] \leavevmode\begin{itemize}
\item {} 
binc: array with bin content

\item {} 
bine: array with bin edges

\end{itemize}

\item[{Returns:}] \leavevmode\begin{itemize}
\item {} 
float: mean, sigma and sigma on mean

\end{itemize}

\end{description}

\end{fulllineitems}

\index{kFit() (in module PhyPraKit.PhyPraKit)}

\begin{fulllineitems}
\phantomsection\label{\detokenize{index:PhyPraKit.PhyPraKit.kFit}}\pysiglinewithargsret{\sphinxcode{PhyPraKit.PhyPraKit.}\sphinxbfcode{kFit}}{\emph{func, x, y, sx, sy, p0=None, p0e=None, xabscor=None, yabscor=None, xrelcor=None, yrelcor=None, title='Daten', axis\_labels={[}'X', `Y'{]}, plot=True, quiet=False}}{}~\begin{quote}

fit function func with errors on x and y;
uses package \sphinxtitleref{kafe}
\begin{description}
\item[{Args:}] \leavevmode\begin{itemize}
\item {} 
func: function to fit

\item {} 
x:  np-array, independent data

\item {} 
y:  np-array, dependent data

\end{itemize}

\item[{the following are single floats or arrays of length of x}] \leavevmode\begin{itemize}
\item {} 
sx: uncertainty(ies) on x

\item {} 
sy: uncertainty(ies) on y

\item {} 
p0: array-like, initial guess of parameters

\item {} 
p0e: array-like, initial guess of parameter uncertainties

\item {} 
xabscor: absolute, correlated error(s) on x

\item {} 
yabscor: absolute, correlated error(s) on y

\item {} 
xrelcor: relative, correlated error(s) on x

\item {} 
yrelcor: relative, correlated error(s) on y

\item {} 
title:   string, title of gaph

\item {} 
axis\_labels: List of strings, axis labels x and y

\item {} 
plot: flag to switch off graphical ouput

\item {} 
quiet: flag to suppress text and log output

\end{itemize}

\end{description}
\end{quote}
\begin{description}
\item[{Returns:}] \leavevmode\begin{itemize}
\item {} 
np-array of float: parameter values

\item {} 
np-array of float: parameter errors

\item {} 
np-array: cor   correlation matrix

\item {} 
float: chi2  chi-square

\end{itemize}

\end{description}

\end{fulllineitems}

\index{kRegression() (in module PhyPraKit.PhyPraKit)}

\begin{fulllineitems}
\phantomsection\label{\detokenize{index:PhyPraKit.PhyPraKit.kRegression}}\pysiglinewithargsret{\sphinxcode{PhyPraKit.PhyPraKit.}\sphinxbfcode{kRegression}}{\emph{x, y, sx, sy, xabscor=None, yabscor=None, xrelcor=None, yrelcor=None, title='Daten', axis\_labels={[}'X', `Y'{]}, plot=True, quiet=False}}{}~\begin{quote}

linear regression y(x) = ax + b  with errors on x and y;
uses package \sphinxtitleref{kafe}
\begin{description}
\item[{Args:}] \leavevmode\begin{itemize}
\item {} 
x:  np-array, independent data

\item {} 
y:  np-array, dependent data

\end{itemize}

\item[{the following are single floats or arrays of length of x}] \leavevmode\begin{itemize}
\item {} 
sx: uncertainty(ies) on x

\item {} 
sy: uncertainty(ies) on y

\item {} 
xabscor: absolute, correlated error(s) on x

\item {} 
yabscor: absolute, correlated error(s) on y

\item {} 
xrelcor: relative, correlated error(s) on x

\item {} 
yrelcor: relative, correlated error(s) on y

\item {} 
title:   string, title of gaph

\item {} 
axis\_labels: List of strings, axis labels x and y

\item {} 
plot: flag to switch off graphical ouput

\item {} 
quiet: flag to suppress text and log output

\end{itemize}

\end{description}
\end{quote}
\begin{description}
\item[{Returns:}] \leavevmode\begin{itemize}
\item {} 
float: a     slope

\item {} 
float: b     constant

\item {} 
float: sa    sigma on slope

\item {} 
float: sb    sigma on constant

\item {} 
float: cor   correlation

\item {} 
float: chi2  chi-square

\end{itemize}

\end{description}

\end{fulllineitems}

\index{labxParser() (in module PhyPraKit.PhyPraKit)}

\begin{fulllineitems}
\phantomsection\label{\detokenize{index:PhyPraKit.PhyPraKit.labxParser}}\pysiglinewithargsret{\sphinxcode{PhyPraKit.PhyPraKit.}\sphinxbfcode{labxParser}}{\emph{file}, \emph{prlevel=1}}{}
read files in xml-format produced with Leybold CASSY
\begin{description}
\item[{Args:}] \leavevmode\begin{itemize}
\item {} 
file:  input data in .labx format

\item {} 
prlevel: control printout level, 0=no printout

\end{itemize}

\item[{Returns:}] \leavevmode\begin{itemize}
\item {} 
list of strings: tags of measurmement vectors

\item {} 
2d list:         measurement vectors read from file

\end{itemize}

\end{description}

\end{fulllineitems}

\index{linRegression() (in module PhyPraKit.PhyPraKit)}

\begin{fulllineitems}
\phantomsection\label{\detokenize{index:PhyPraKit.PhyPraKit.linRegression}}\pysiglinewithargsret{\sphinxcode{PhyPraKit.PhyPraKit.}\sphinxbfcode{linRegression}}{\emph{x}, \emph{y}, \emph{sy}}{}
linear regression y(x) = ax + b
\begin{description}
\item[{method: }] \leavevmode
analytical formula

\item[{Args:}] \leavevmode\begin{itemize}
\item {} 
x: np-array, independent data

\item {} 
y: np-array, dependent data

\item {} 
sx: np-array, uncertainty on y

\end{itemize}

\item[{Returns:}] \leavevmode\begin{itemize}
\item {} 
float: a     slope

\item {} 
float: b     constant

\item {} 
float: sa  sigma on slope

\item {} 
float: sb  sigma on constant

\item {} 
float: cor   correlation

\item {} 
float: chi2  chi-square

\end{itemize}

\end{description}

\end{fulllineitems}

\index{linRegressionXY() (in module PhyPraKit.PhyPraKit)}

\begin{fulllineitems}
\phantomsection\label{\detokenize{index:PhyPraKit.PhyPraKit.linRegressionXY}}\pysiglinewithargsret{\sphinxcode{PhyPraKit.PhyPraKit.}\sphinxbfcode{linRegressionXY}}{\emph{x}, \emph{y}, \emph{sx}, \emph{sy}}{}
linear regression y(x) = ax + b  with errors on x and y
uses numerical \sphinxquotedblleft{}orthogonal distance regression\sphinxquotedblright{} from package scipy.odr
\begin{description}
\item[{Args:}] \leavevmode\begin{itemize}
\item {} 
x:  np-array, independent data

\item {} 
y:  np-array, dependent data

\item {} 
sx: np-array, uncertainty on y

\item {} 
sy: np-array, uncertainty on y

\end{itemize}

\item[{Returns:}] \leavevmode\begin{itemize}
\item {} 
float: a     slope

\item {} 
float: b     constant

\item {} 
float: sa    sigma on slope

\item {} 
float: sb    sigma on constant

\item {} 
float: cor   correlation

\item {} 
float: chi2  chi-square

\end{itemize}

\end{description}

\end{fulllineitems}

\index{meanFilter() (in module PhyPraKit.PhyPraKit)}

\begin{fulllineitems}
\phantomsection\label{\detokenize{index:PhyPraKit.PhyPraKit.meanFilter}}\pysiglinewithargsret{\sphinxcode{PhyPraKit.PhyPraKit.}\sphinxbfcode{meanFilter}}{\emph{a}, \emph{width=5}}{}
apply a sliding average to smoothen data,
\begin{description}
\item[{method:}] \leavevmode
value at index i and int(width/2) neighbours are averaged
to from the new value at index i
\begin{description}
\item[{Args:}] \leavevmode\begin{itemize}
\item {} 
a: np-array of values

\item {} 
width: int, number of points to average over
(if width is an even number, width+1 is used)

\end{itemize}

\item[{Returns:}] \leavevmode\begin{itemize}
\item {} 
av  smoothed signal curve

\end{itemize}

\end{description}

\end{description}

\end{fulllineitems}

\index{nhist() (in module PhyPraKit.PhyPraKit)}

\begin{fulllineitems}
\phantomsection\label{\detokenize{index:PhyPraKit.PhyPraKit.nhist}}\pysiglinewithargsret{\sphinxcode{PhyPraKit.PhyPraKit.}\sphinxbfcode{nhist}}{\emph{data}, \emph{bins=50}, \emph{xlabel='x'}, \emph{ylabel='frequency'}}{}
Histogram.hist
show a one-dimensional histogram
\begin{description}
\item[{Args:}] \leavevmode\begin{itemize}
\item {} 
data: array containing float values to be histogrammed

\item {} 
bins: number of bins

\item {} 
xlabel: label for x-axis

\item {} 
ylabel: label for y axix

\end{itemize}

\item[{Returns:}] \leavevmode\begin{itemize}
\item {} 
float arrays bin content and bin edges

\end{itemize}

\end{description}

\end{fulllineitems}

\index{nhist2d() (in module PhyPraKit.PhyPraKit)}

\begin{fulllineitems}
\phantomsection\label{\detokenize{index:PhyPraKit.PhyPraKit.nhist2d}}\pysiglinewithargsret{\sphinxcode{PhyPraKit.PhyPraKit.}\sphinxbfcode{nhist2d}}{\emph{x}, \emph{y}, \emph{bins=10}, \emph{xlabel='x axis'}, \emph{ylabel='y axis'}, \emph{clabel='counts'}}{}
Histrogram.hist2d
create and plot a 2-dimensional histogram
\begin{description}
\item[{Args:}] \leavevmode\begin{itemize}
\item {} 
x: array containing x values to be histogrammed

\item {} 
y: array containing y values to be histogrammed

\item {} 
bins: number of bins

\item {} 
xlabel: label for x-axis

\item {} 
ylabel: label for y axix

\item {} 
clabel: label for colour index

\end{itemize}

\item[{Returns:}] \leavevmode\begin{itemize}
\item {} 
float array: array with counts per bin

\item {} 
float array: histogram edges in x

\item {} 
float array: histogram edges in y

\end{itemize}

\end{description}

\end{fulllineitems}

\index{odFit() (in module PhyPraKit.PhyPraKit)}

\begin{fulllineitems}
\phantomsection\label{\detokenize{index:PhyPraKit.PhyPraKit.odFit}}\pysiglinewithargsret{\sphinxcode{PhyPraKit.PhyPraKit.}\sphinxbfcode{odFit}}{\emph{fitf}, \emph{x}, \emph{y}, \emph{sx}, \emph{sy}, \emph{p0=None}}{}~\begin{quote}

fit an arbitrary function with errors on x and y
uses numerical \sphinxquotedblleft{}orthogonal distance regression\sphinxquotedblright{} from package scipy.odr
\begin{description}
\item[{Args:}] \leavevmode\begin{itemize}
\item {} 
fitf: function to fit, arguments (array:P, float:x)

\item {} 
x:  np-array, independent data

\item {} 
y:  np-array, dependent data

\item {} 
sx: np-array, uncertainty on x

\item {} 
sy: np-array, uncertainty on y

\item {} 
p0: none, scalar or array, initial guess of parameters

\end{itemize}

\end{description}
\end{quote}
\begin{description}
\item[{Returns:}] \leavevmode\begin{itemize}
\item {} 
np-array of float: parameter values

\item {} 
np-array of float: parameter errors

\item {} 
np-array: cor   correlation matrix

\item {} 
float: chi2  chi-square

\end{itemize}

\end{description}

\end{fulllineitems}

\index{offsetFilter() (in module PhyPraKit.PhyPraKit)}

\begin{fulllineitems}
\phantomsection\label{\detokenize{index:PhyPraKit.PhyPraKit.offsetFilter}}\pysiglinewithargsret{\sphinxcode{PhyPraKit.PhyPraKit.}\sphinxbfcode{offsetFilter}}{\emph{a}}{}
correct an offset in array a 
(assuming a symmetric signal around zero)
by subtracting the mean

\end{fulllineitems}

\index{profile2d() (in module PhyPraKit.PhyPraKit)}

\begin{fulllineitems}
\phantomsection\label{\detokenize{index:PhyPraKit.PhyPraKit.profile2d}}\pysiglinewithargsret{\sphinxcode{PhyPraKit.PhyPraKit.}\sphinxbfcode{profile2d}}{\emph{H2d}, \emph{xed}, \emph{yed}}{}~\begin{description}
\item[{generate a profile plot from 2d histogram:}] \leavevmode\begin{itemize}
\item {} 
mean y at a centre of x-bins, standard deviations as error bars

\end{itemize}

\item[{Args:}] \leavevmode\begin{itemize}
\item {} 
H2d: histogram array (as returned by histogram2d)

\item {} 
xed: bin edges in x

\item {} 
yed: bin edges in y

\end{itemize}

\item[{Returns:}] \leavevmode\begin{itemize}
\item {} 
float: array of bin centres in x

\item {} 
float: array mean

\item {} 
float: array rms

\item {} 
float: array sigma on mean

\end{itemize}

\end{description}

\end{fulllineitems}

\index{readCSV() (in module PhyPraKit.PhyPraKit)}

\begin{fulllineitems}
\phantomsection\label{\detokenize{index:PhyPraKit.PhyPraKit.readCSV}}\pysiglinewithargsret{\sphinxcode{PhyPraKit.PhyPraKit.}\sphinxbfcode{readCSV}}{\emph{file}, \emph{nlhead=1}}{}
read Data in .csv format, skip header lines
\begin{description}
\item[{Args:}] \leavevmode\begin{itemize}
\item {} 
file: string, file name

\item {} 
nhead: number of header lines to skip

\item {} 
delim: column separator

\end{itemize}

\item[{Returns:}] \leavevmode\begin{itemize}
\item {} 
hlines: list of string, header lines

\item {} 
data: 2d array, 1st index for columns

\end{itemize}

\end{description}

\end{fulllineitems}

\index{readCassy() (in module PhyPraKit.PhyPraKit)}

\begin{fulllineitems}
\phantomsection\label{\detokenize{index:PhyPraKit.PhyPraKit.readCassy}}\pysiglinewithargsret{\sphinxcode{PhyPraKit.PhyPraKit.}\sphinxbfcode{readCassy}}{\emph{file}, \emph{prlevel=0}}{}
read Data exported from Cassy in .txt format
\begin{description}
\item[{Args:}] \leavevmode\begin{itemize}
\item {} 
file: string, file name

\item {} 
prlevel: printout level, 0 means silent

\end{itemize}

\item[{Returns:}] \leavevmode\begin{itemize}
\item {} 
units: list of strings, channel units

\item {} 
data: tuple of arrays, channel data

\end{itemize}

\end{description}

\end{fulllineitems}

\index{readColumnData() (in module PhyPraKit.PhyPraKit)}

\begin{fulllineitems}
\phantomsection\label{\detokenize{index:PhyPraKit.PhyPraKit.readColumnData}}\pysiglinewithargsret{\sphinxcode{PhyPraKit.PhyPraKit.}\sphinxbfcode{readColumnData}}{\emph{fname}, \emph{cchar='\#'}, \emph{delimiter=None}, \emph{pr=True}}{}~\begin{description}
\item[{read column-data from file}] \leavevmode\begin{itemize}
\item {} 
input is assumed to be columns of floats

\item {} 
characters following \textless{}cchar\textgreater{}, and \textless{}cchar\textgreater{} itself, are ignored

\item {} 
words with preceeding `*' are taken as keywords for meta-data,
text following the keyword is returned in a dictionary

\end{itemize}

\item[{Args:}] \leavevmode\begin{itemize}
\item {} 
string fnam:      file name

\item {} 
int ncols:        number of columns

\item {} 
char delimiter:   character separating columns

\item {} 
bool pr:          print input to std out if True

\end{itemize}

\end{description}

\end{fulllineitems}

\index{readPicoScope() (in module PhyPraKit.PhyPraKit)}

\begin{fulllineitems}
\phantomsection\label{\detokenize{index:PhyPraKit.PhyPraKit.readPicoScope}}\pysiglinewithargsret{\sphinxcode{PhyPraKit.PhyPraKit.}\sphinxbfcode{readPicoScope}}{\emph{file}, \emph{prlevel=0}}{}
read Data exported from PicoScope in .txt or .csv format
\begin{description}
\item[{Args:}] \leavevmode\begin{itemize}
\item {} 
file: string, file name

\item {} 
prlevel: printout level, 0 means silent

\end{itemize}

\item[{Returns:}] \leavevmode\begin{itemize}
\item {} 
units: list of strings, channel units

\item {} 
data: tuple of arrays, channel data

\end{itemize}

\end{description}

\end{fulllineitems}

\index{readtxt() (in module PhyPraKit.PhyPraKit)}

\begin{fulllineitems}
\phantomsection\label{\detokenize{index:PhyPraKit.PhyPraKit.readtxt}}\pysiglinewithargsret{\sphinxcode{PhyPraKit.PhyPraKit.}\sphinxbfcode{readtxt}}{\emph{file}, \emph{nlhead=1}, \emph{delim='\textbackslash{}t'}}{}~\begin{description}
\item[{read floating point data in general txt format}] \leavevmode
skip header lines, replace decimal comma, remove special characters

\item[{Args:}] \leavevmode\begin{itemize}
\item {} 
file: string, file name

\item {} 
nhead: number of header lines to skip

\item {} 
delim: column separator

\end{itemize}

\item[{Returns:}] \leavevmode\begin{itemize}
\item {} 
hlines: list of string, header lines

\item {} 
data: 2d array, 1st index for columns

\end{itemize}

\end{description}

\end{fulllineitems}

\index{resample() (in module PhyPraKit.PhyPraKit)}

\begin{fulllineitems}
\phantomsection\label{\detokenize{index:PhyPraKit.PhyPraKit.resample}}\pysiglinewithargsret{\sphinxcode{PhyPraKit.PhyPraKit.}\sphinxbfcode{resample}}{\emph{a}, \emph{t=None}, \emph{n=11}}{}
perform average over n data points of array a, 
return reduced array, eventually with corresponding time values
\begin{description}
\item[{method:}] \leavevmode
value at index \sphinxtitleref{i} and \sphinxtitleref{int(width/2)} neighbours are averaged
to from the new value at index \sphinxtitleref{i}
\begin{description}
\item[{Args:}] \leavevmode\begin{itemize}
\item {} 
a, t: np-arrays of values of same length

\item {} 
width: int, number of values of array \sphinxtitleref{a} to average over
(if width is an even number, width+1 is used)

\end{itemize}

\item[{Returns:}] \leavevmode\begin{itemize}
\item {} 
av: array with reduced number of samples

\item {} 
tav:  a second, related array with reduced number of samples

\end{itemize}

\end{description}

\end{description}

\end{fulllineitems}

\index{simplePeakfinder() (in module PhyPraKit.PhyPraKit)}

\begin{fulllineitems}
\phantomsection\label{\detokenize{index:PhyPraKit.PhyPraKit.simplePeakfinder}}\pysiglinewithargsret{\sphinxcode{PhyPraKit.PhyPraKit.}\sphinxbfcode{simplePeakfinder}}{\emph{x}, \emph{a}, \emph{th=0.0}}{}~\begin{description}
\item[{find positions of all Peaks and Dips in data}] \leavevmode
x-coordinates are determined from weighted average over 3 data points

\end{description}

this only works for very smooth data with well defined extrema
use \sphinxcode{convolutionPeakfinder} instead
\begin{quote}
\begin{description}
\item[{Args:}] \leavevmode\begin{itemize}
\item {} 
x: np-array of values

\item {} 
a: np-array of values at position x

\item {} 
th: float, threshold for peaks

\end{itemize}

\item[{Returns:}] \leavevmode\begin{itemize}
\item {} 
np-array: x positions of peaks as weighted mean over neighbours

\item {} 
np-array: y positions

\end{itemize}

\end{description}
\end{quote}

\end{fulllineitems}

\index{smearData() (in module PhyPraKit.PhyPraKit)}

\begin{fulllineitems}
\phantomsection\label{\detokenize{index:PhyPraKit.PhyPraKit.smearData}}\pysiglinewithargsret{\sphinxcode{PhyPraKit.PhyPraKit.}\sphinxbfcode{smearData}}{\emph{d}, \emph{s}, \emph{srel=None}, \emph{abscor=None}, \emph{relcor=None}}{}~\begin{description}
\item[{Generate measurement data from \sphinxquotedblleft{}true\sphinxquotedblright{} input d by}] \leavevmode
adding random deviations according to the uncertainties

\item[{Args:}] \leavevmode\begin{itemize}
\item {} 
d:  np-array, (true) input data

\end{itemize}

\item[{the following are single floats or arrays of length of array d}] \leavevmode\begin{itemize}
\item {} 
s: gaussian uncertainty(ies) (absolute)

\item {} 
srel: gaussian uncertainties (relative)

\end{itemize}

\item[{the following are common (correlated) systematic uncertainties}] \leavevmode\begin{itemize}
\item {} 
abscor: absolute, correlated uncertainty

\item {} 
relcor: relative, correlated uncertainty

\end{itemize}

\item[{Returns:}] \leavevmode\begin{itemize}
\item {} 
np-array of floats: dm, smeared (=measured) data

\end{itemize}

\end{description}

\end{fulllineitems}

\index{wmean() (in module PhyPraKit.PhyPraKit)}

\begin{fulllineitems}
\phantomsection\label{\detokenize{index:PhyPraKit.PhyPraKit.wmean}}\pysiglinewithargsret{\sphinxcode{PhyPraKit.PhyPraKit.}\sphinxbfcode{wmean}}{\emph{x}, \emph{sx}, \emph{pr=True}}{}
weighted mean of np-array x with uncertainties sx
\begin{description}
\item[{Args:}] \leavevmode\begin{itemize}
\item {} 
x: np-array of values

\item {} 
sx: uncertainties

\item {} 
pr: if True, print result

\end{itemize}

\item[{Returns:}] \leavevmode\begin{itemize}
\item {} 
float: mean, sigma

\end{itemize}

\end{description}

\end{fulllineitems}

\index{writeCSV() (in module PhyPraKit.PhyPraKit)}

\begin{fulllineitems}
\phantomsection\label{\detokenize{index:PhyPraKit.PhyPraKit.writeCSV}}\pysiglinewithargsret{\sphinxcode{PhyPraKit.PhyPraKit.}\sphinxbfcode{writeCSV}}{\emph{file}, \emph{ldata}, \emph{hlines={[}{]}}, \emph{fmt='\%.10g'}}{}
write data in .csv format, including header lines
\begin{description}
\item[{Args:}] \leavevmode\begin{itemize}
\item {} 
file: string, file name

\item {} 
ldata: list of columns to be written

\item {} 
hlines: list with header lines (optional)

\item {} 
fmt: format string (optional)

\end{itemize}

\item[{Returns: }] \leavevmode\begin{itemize}
\item {} 
0/1  for success/fail

\end{itemize}

\end{description}

\end{fulllineitems}

\phantomsection\label{\detokenize{index:module-test_readColumnData}}\index{test\_readColumnData (module)}
test\_readColumnData.py
\begin{quote}

test data input from text file with module PhyPraKit.readColumnData
\end{quote}
\phantomsection\label{\detokenize{index:module-test_readtxt}}\index{test\_readtxt (module)}
test\_readtxt.py
\begin{quote}

uses readtxt() to read floating-point column-data in general .txt format

example output from PicoTech 8 channel data logger,
with `       ` separated values, 2 header lines,
german decimal comma and special character `\textasciicircum{}@'
\end{quote}
\phantomsection\label{\detokenize{index:module-test_readPicoScope}}\index{test\_readPicoScope (module)}
test\_readPicoSocpe.py
\begin{quote}

Einlesen von Daten aus einer mit PicoScope erstellten Datei
\end{quote}
\phantomsection\label{\detokenize{index:module-test_labxParser}}\index{test\_labxParser (module)}
cassylabxParser.py
\begin{quote}

read files in xml-format produced with the Leybold Cassy system

uses PhyPraPit.labxParser()
\end{quote}
\phantomsection\label{\detokenize{index:module-test_Histogram}}\index{test\_Histogram (module)}
test\_Historgram.py
\begin{quote}

test histogram functionality in PhyPraKit
\end{quote}
\phantomsection\label{\detokenize{index:module-test_AutoCorrelation}}\index{test\_AutoCorrelation (module)}
test\_AutoCorrelation.py
\begin{quote}

test function \sphinxtitleref{autocorrelate()} in PhyPraKit; 
determines the frequency of a periodic signal from maxima and minima
of the autocorrelation function and performs statistical analysis
of time between peaks/dips

uses \sphinxtitleref{readCSV()}, \sphinxtitleref{autocorrelate()}, \sphinxtitleref{convolutionPeakfinder()} 
and \sphinxtitleref{histstat()} from PhyPraKit
\end{quote}
\phantomsection\label{\detokenize{index:module-test_kRegression}}\index{test\_kRegression (module)}\begin{description}
\item[{test\_kRegression}] \leavevmode
test linear regression with kafe using kFit from PhyPrakKit
uncertainties in x and y and correlated 
absolute and relative uncertainties

\end{description}
\phantomsection\label{\detokenize{index:module-test_kFit}}\index{test\_kFit (module)}\begin{description}
\item[{test\_kFit}] \leavevmode
test fiting an arbitrary fucntion with kafe, 
with uncertainties in x and y and correlated 
absolute and relative uncertainties

\end{description}
\phantomsection\label{\detokenize{index:module-test_generateData}}\index{test\_generateData (module)}
test\_generateDate
\begin{quote}

test generation of simulated data
this simulates a measurement with given x-values with uncertainties;
random deviations are then added to arrive at the true values, from
which the true y-values are then calculated according to a model
function. In the last step, these true y-values are smeared
by adding random deviations to obtain a sample of measured values
\end{quote}
\phantomsection\label{\detokenize{index:module-kfitf}}\index{kfitf (module)}
kfitf.py

Perform a fit with the kafe package driven by input file

usage: kfitf.py {[}-h{]} {[}-n{]} {[}-s{]} {[}-c{]} {[}--noinfo{]} {[}-f FORMAT{]} filename
\begin{description}
\item[{positional arguments:}] \leavevmode
filename                       name of fit input file

\item[{optional arguments:}] \leavevmode\begin{optionlist}{3cm}
\item [-h, -{-}help]  
show this help message and exit
\item [-n, -{-}noplot]  
suppress ouput of plots on screen
\item [-s, -{-}saveplot]  
save plot(s) in file(s)
\item [-c, -{-}contour]  
plot contours and profiles
\item [-{-}noinfo]  
suppress fit info on plot
\item [-{-}noband]  
suppress 1-sigma band around function
\item [-{-}format FMT]  
graphics output format, default FMT = pdf
\end{optionlist}

\end{description}
\phantomsection\label{\detokenize{index:module-Beispiel_Drehpendel}}\index{Beispiel\_Drehpendel (module)}
Beispliel\_Drehpendel.py
\begin{quote}

Auswertung der Daten aus einer im CASSY labx-Format gespeicherten Datei
am Beispiel des Drehpendels
\begin{itemize}
\item {} 
Einlesen der Daten im .labx-Format

\item {} 
Säubern der Daten durch verschiedene Filterfunktionen:
- offset-Korrektur
- Glättung durch gleitenden Mittelwert
- Zusammenfassung benachberter Daten durch Mittelung

\item {} 
Fourier-Transformation (einfach und fft)

\item {} 
Suche nach Extrema (\sphinxtitleref{peaks} und \sphinxtitleref{dips})

\item {} 
Anpassung von Funkionen an Einhüllende der Maxima und Minima

\item {} 
Interpolation durch Spline-Funktionen

\item {} 
numerische Ableitung und Ableitung der Splines

\item {} 
Phasenraum-Darstellung (aufgezeichnete Wellenfunktion
gegen deren Ableitung nach der Zeit)

\end{itemize}
\end{quote}
\phantomsection\label{\detokenize{index:module-Beispiel_Hysterese}}\index{Beispiel\_Hysterese (module)}
Beispiel\_Hysterese.py
\begin{quote}

Auswertung der Daten aus einer mit PicoScope erstellten Datei
im txt-Format am Beispiel des Hystereseversuchs
\begin{itemize}
\item {} 
Einlesen der Daten aus PicoScope-Datei vom Typ .txt oder .csv

\item {} 
Darstellung  Kanal\_a vs. Kanal\_b

\item {} 
Auftrennung in zwei Zweige für steigenden bzw. abnehmenden Strom

\item {} 
Interpolation durch kubische Splines

\item {} 
Integration der Spline-Funktionen

\end{itemize}
\end{quote}
\phantomsection\label{\detokenize{index:module-Beispiel_Wellenform}}\index{Beispiel\_Wellenform (module)}
test\_readPicoSocpe.py
\begin{quote}

Einlesen von Daten aus dem mit PicoScope erstellten Dateien
am Beispiel der akustischen Anregung eines Stabes
\end{quote}
\phantomsection\label{\detokenize{index:module-Beispiel_GammaSpektroskopie}}\index{Beispiel\_GammaSpektroskopie (module)}
Beispliel\_Drehpendel.py
\begin{quote}

Darstellung der Daten aus einer im CASSY labx-Format gespeicherten Datei
am Beispiel der Gamma-Spektroskopie
\begin{itemize}
\item {} 
Einlesen der Daten im .labx-Format

\end{itemize}
\end{quote}


\renewcommand{\indexname}{Python Module Index}
\begin{sphinxtheindex}
\def\bigletter#1{{\Large\sffamily#1}\nopagebreak\vspace{1mm}}
\bigletter{b}
\item {\sphinxstyleindexentry{Beispiel\_Drehpendel}}\sphinxstyleindexpageref{index:\detokenize{module-Beispiel_Drehpendel}}
\item {\sphinxstyleindexentry{Beispiel\_GammaSpektroskopie}}\sphinxstyleindexpageref{index:\detokenize{module-Beispiel_GammaSpektroskopie}}
\item {\sphinxstyleindexentry{Beispiel\_Hysterese}}\sphinxstyleindexpageref{index:\detokenize{module-Beispiel_Hysterese}}
\item {\sphinxstyleindexentry{Beispiel\_Wellenform}}\sphinxstyleindexpageref{index:\detokenize{module-Beispiel_Wellenform}}
\indexspace
\bigletter{k}
\item {\sphinxstyleindexentry{kfitf}}\sphinxstyleindexpageref{index:\detokenize{module-kfitf}}
\indexspace
\bigletter{p}
\item {\sphinxstyleindexentry{PhyPraKit}}\sphinxstyleindexpageref{index:\detokenize{module-PhyPraKit}}
\item {\sphinxstyleindexentry{PhyPraKit.PhyPraKit}}\sphinxstyleindexpageref{index:\detokenize{module-PhyPraKit.PhyPraKit}}
\indexspace
\bigletter{t}
\item {\sphinxstyleindexentry{test\_AutoCorrelation}}\sphinxstyleindexpageref{index:\detokenize{module-test_AutoCorrelation}}
\item {\sphinxstyleindexentry{test\_generateData}}\sphinxstyleindexpageref{index:\detokenize{module-test_generateData}}
\item {\sphinxstyleindexentry{test\_Histogram}}\sphinxstyleindexpageref{index:\detokenize{module-test_Histogram}}
\item {\sphinxstyleindexentry{test\_kFit}}\sphinxstyleindexpageref{index:\detokenize{module-test_kFit}}
\item {\sphinxstyleindexentry{test\_kRegression}}\sphinxstyleindexpageref{index:\detokenize{module-test_kRegression}}
\item {\sphinxstyleindexentry{test\_labxParser}}\sphinxstyleindexpageref{index:\detokenize{module-test_labxParser}}
\item {\sphinxstyleindexentry{test\_readColumnData}}\sphinxstyleindexpageref{index:\detokenize{module-test_readColumnData}}
\item {\sphinxstyleindexentry{test\_readPicoScope}}\sphinxstyleindexpageref{index:\detokenize{module-test_readPicoScope}}
\item {\sphinxstyleindexentry{test\_readtxt}}\sphinxstyleindexpageref{index:\detokenize{module-test_readtxt}}
\end{sphinxtheindex}

\renewcommand{\indexname}{Index}
\printindex
\end{document}